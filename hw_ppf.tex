% --- PDF/UA metadata (new + fallback) ---
\ExplSyntaxOn
\keys_if_exist:nnTF { document / metadata } { tags }
  {
    \DocumentMetadata{
      pdfstandard = ua-1,
      pdfversion  = 1.7,
      tags,
      lang        = en-US,
      uncompress
    }
  }
  {
    \RequirePackage{pdfmanagement-testphase}
    \DocumentMetadata{
      testphase   = { tagging, uncompress },
      pdfstandard = ua-1,
      pdfversion  = 1.7,
      lang        = en-US
    }
  }
\ExplSyntaxOff
%%% ----------------- CHANGE THE DOCUMENT CLASS HERE! ----------------- %%%
% \documentclass[ignorenonframetext]{beamer}
% \usepackage{beamerthemeshadow}

% OR %
\documentclass{article}

%%% ------------------------------------------------------------------- %%%

% Load tagpdf and activate tagging
\RequirePackage{tagpdf}
\tagpdfsetup{activate, uncompress, interwordspace=true}

\usepackage{fontspec}
\usepackage[english]{babel}
\usepackage{cmap}

\usepackage{hyperref}
\hypersetup{
  pdfencoding=auto,
  pdflang={en-US},
  %%% ----------------- CHANGE TITLE HERE! ----------------- %%%
  pdftitle={Homework: Production Possibilities Frontiers},     
  %%% ------------------------------------------------------ %%%
  pdfdisplaydoctitle=true,
  colorlinks=true,
  urlcolor=red
} 

\usepackage{bookmark}
\bookmarksetup{open,numbered}

%% --------- END OF TEMPLATE FOR ACCESSIBILITY --------- %%

\usepackage[T1]{fontenc}
\usepackage{calc}
\usepackage{setspace}
\usepackage{multicol}
\usepackage{enumitem}

\usepackage{graphicx}
\usepackage{color}

\setlength{\voffset}{-0.25in}
\setlength{\topmargin}{0pt}
\setlength{\hoffset}{-0.25in}
\setlength{\oddsidemargin}{0pt}
\setlength{\headheight}{0pt}
\setlength{\headsep}{.1in}
\setlength{\marginparsep}{0pt}
\setlength{\marginparwidth}{0pt}
\setlength{\marginparpush}{0pt}
\setlength{\footskip}{.4in}
\setlength{\textwidth}{7in}
\setlength{\textheight}{9.5in}
\setlength{\parskip}{1pc}
\setlength{\parindent}{0pc}

\renewcommand{\baselinestretch}{1.1}
\newcommand{\ds}{\displaystyle}

\begin{document}
\thispagestyle{empty}

\noindent \textbf{ECO 120: Global Macroeconomics} \\
\noindent \textbf{Week 1 Homework: Production Possibilities Frontiers}

\noindent \textbf{Directions:} Notice that all problems ask you for a descriptive answer in addition to performing a calculation or manipulating a graphical model. For full credit, make sure that you do both. You may print these sheets and put your answers in the space provided or you may use your own paper to write your answers.

When you have finished, scan or take pictures of your work, combine all images to a single PDF file, and upload your work as a single PDF file to the Canvas Assignment area. There are apps available for Apple and Android mobile devices that can create PDF documents using your device's camera, including the Apple iPhone's native \textit{Notes} app (use the \textit{scan document} feature) and \textit{Adobe Scan} app available for Android and Apple mobile devices. There are also free online tools such as \href{https://online2pdf.com/}{Online2PDF} and \href{https://www.easypdfcloud.com/}{Easy PDF Cloud}.

You may print this document and answer the questions in the space provided or you may use your own paper.

\textbf{Scenario:} The great hypothetical sovereign nation Farmland can grow corn to be used for food or to be used for ethanol.  The following table presents hypothetical production possibilities when all resources are used efficiently. \\

\begin{tabular}{|p{0.8in}|c|c|}\hline
Production Choice & Corn (food) & Ethanol \\ \hline
A & 30 & 0 \\
B & 27 & 3 \\
C & 21 & 6 \\
D & 12 & 9 \\
E & 0 & 12 \\ \hline
\end{tabular}

\begin{enumerate}
\item Draw the production possibilities for corn (food) and ethanol.  Label the parts of the graph that are attainable but not efficient, most efficient, and not attainable.
\newpage
\item Compute the opportunity cost of ethanol for each level of production in the table.  What happens to the opportunity cost of ethanol as the production of ethanol increases?
\vspace*{2.5in}
\item Describe and illustrate the impact on the production possibilities curve if there was a drought.
\vspace*{2.5in}
\item Describe and illustrate the impact on the production possibilities curve if governments stop encouraging the use of biofuels such as ethanol.
\end{enumerate}
\newpage

Read the following article posted on Canvas: Roger Thurow, "Makeshift Cuisinart Makes a Lot Possible in Impoverished Mali", \textit{The Wall Street Journal}, July 26, 2002, and answer the following questions

\begin{enumerate}
\setcounter{enumi}{4}
\item Graph a production possibilities frontier with two categories of goods. Let one good be peanut butter and all other products that are made by grinding nuts, beans, seeds, etc. Let the other category be "All other goods," which would include absolutely anything else, including other crops, clothing, electricity, education, automobiles, etc.  Graph the PPF so that it follows the law of increasing opportunity costs.
\vspace*{2in}
\item The graph shows numerous possibilities for what Mali villages can produce. Keep in mind that Mali is an impoverished country (so people have very little "all other goods") and nearly half of the population is involved in the production of peanut butter (and similar blended products). Choose and label a \textit{point} on the PPF above where you think they are likely producing before the invention is introduced.
\vspace*{2pc}

\item Redraw the PPF from Question 5 and on this same graph, show how the PPF may change immediately following the introduction of the invention that allows women to grind peanuts.
\vspace*{2in}

\newpage

\item After the introduction of the invention, villages in Mali were able to enjoy more goods than just more peanut butter.  Name at least 3 of these goods.
\vspace*{1in}

\item As a result of the peanut grinding invention, do you think there was a greater increase in production of peanut butter or production of all other goods? Think about the focus of the article. Redraw your PPF graph in Question 7, by suggesting and labeling a point on the old PPF where the villages were producing before the invention, and suggest and label a point on the new PPF where you believe the villages were producing after the invention.

\vspace*{2.5in}

\item As time passed after the introduction of the invention, many villages decided to reallocate scarce resources towards technology, capital (electricity, lights in hospitals, etc), and literacy.  What distinguishes these scarce resources compared to other goods? Redraw the production possibilities frontier from Question 7 and show what has happens to the PPF as a result of these additional investments.


\end{enumerate}



\end{document}

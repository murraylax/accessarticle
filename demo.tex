% --- PDF/UA metadata (new + fallback) ---
\ExplSyntaxOn
\keys_if_exist:nnTF { document / metadata } { tags }
  {
    \DocumentMetadata{
      pdfstandard = ua-1,
      pdfversion  = 1.7,
      tags,
      lang        = en-US,
      uncompress
    }
  }
  {
    \RequirePackage{pdfmanagement-testphase}
    \DocumentMetadata{
      testphase   = { tagging, uncompress },
      pdfstandard = ua-1,
      pdfversion  = 1.7,
      lang        = en-US
    }
  }
\ExplSyntaxOff

\documentclass{accessarticle}

% Adjust the margins or other aspects of the page geometry as desired
\usepackage[margin=1in]{geometry}


% Title is required, whether you use it for \maketitle or not
% It sets the PDF title metadata
\title{Accessible Article Template}

\author{James M. Murray}
\date{\today}

\begin{document}

\maketitle
\begin{abstract}
  This is a demonstration of an accessible article template using the \texttt{accessarticle} class.
\end{abstract}

\section{Introduction}

This accessible article template is designed to help authors create documents that comply with UA-1 PDF accessibility standards. It includes features for tagging content, providing alternative text for images, and properly marking up mathematical expressions.

Documents created with the \texttt{accessarticle} document class must be built with \texttt{lualatex}. Tex engines such as \texttt{pdflatex} or \texttt{xelatex} will not work properly with the accessibility features and will result in build errors.

\section{Including Graphics}
\subsection{Accessible Graphics with Alt Text}

The \texttt{includegraphics} environment is redefined to require alternative text for accessibility. Here is an example:\\

\includegraphics[width=0.7\textwidth]
  {Production possibilities frontier (PPF) illustrating that points inside the PPF are possible but inefficient, points on the frontier and possible and efficient, and points outside are not possible.}
  {./myimage.jpg}

\subsection{Purely Decorative Graphics}

Use the new \texttt{decographics} command for purely decorative images that should be ignored by assistive technologies and therefore do not use alternative text:\\

\decographics[width=0.5\textwidth]{./decorative.jpg}

\section{Equations}

The \texttt{displaymath} environment is enhanced to optionally provide alt text for your math expressions. \\
  
\begin{displaymath}[Sample standard deviation]
  s = \sqrt{\frac{\displaystyle\sum_{i=1}^{n} x_i^2 - \frac{1}{n} \left(\sum_{i=1}^{n} x_i\right)^2}{n - 1}}
\end{displaymath}

\ \\ \ \\
The \texttt{equation} environment is similarly enhanced: 

\begin{equation}[Euler's Identity]
  e^{i\pi} + 1 = 0 
\end{equation}

\section{Conclusion}

This document class is intended to be simple with minimal additional features so that \LaTeX articles can meet accessibility standards.  

Copyright 2025 James Murray, \href{https://www.gnu.org/licenses/gpl-3.0.en.html}{GNU General Public License v3.0}

This work is distributed in the hope that it will be useful, but WITHOUT ANY WARRANTY; without even the implied warranty of FITNESS FOR A PARTICULAR PURPOSE.

I welcome bug reports, feedback, and suggestions, but I have very limited time to maintain this project. Do not expect responses, fixes, or new features in a timely manner.


\end{document}
